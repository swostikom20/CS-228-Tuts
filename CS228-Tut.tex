\documentclass{article}
\usepackage[utf8]{inputenc}
\usepackage{a4wide}
\usepackage{dsfont}
\usepackage{upgreek}
\usepackage{setspace} \doublespacing
\usepackage{mathtools}
\usepackage{titlesec}
\usepackage[]{biblatex}
\usepackage{xcolor}
\usepackage{graphicx}
\usepackage[]{amsmath,amssymb,amsfonts, amsthm}
\usepackage{geometry}
\usepackage{breqn}
\usepackage{multicol}
\usepackage{tikz}
\usepackage{mathdots}
\usepackage{color}
\usetikzlibrary{fadings}
\usetikzlibrary{patterns}
\usetikzlibrary{shadows.blur}
\usetikzlibrary{shapes}
\title{Tutorial-7}
\author{Om Swostik}
\date{}
\begin{document}
\maketitle
\begin{flushleft}
\textbf{15.9(ii)}: \\
Asssume for contradiction there exists a FO sentence $F$ without $=$ such that F doesn't satisfy any model with $> 2$ elements.\\
Let $m$ be a model over $\{a,b\}$ such that $m \models F$. (with any signature)\\
Construct a model $m^{'}$ (over the same signature) such that $D_{m^{'}}=\{a,b,c\}$ and $m^{'}(x)=m(x)$ ($\forall x\in FVars$)\\
And $m^{'}(c)=a$, $m^{'}$ is identity over $\{a,b\}$.\\
If $f\in R$, $f_{m^{'}}(l_1,\dots,l_n)$ is true if $f_m(m^{'}(l_1),\dots,m^{'}(l_n))$ holds. ($l_i \in \{a,b,c\}$)\\
For $g\in F$, $g_{m^{'}}(l_1,\dots,l_n)= g_m(m^{'}(l_1),\dots,m^{'}(l_n))$ ($l_i \in \{a,b,c\}$) \\
(Basically replace all occurences of c with a)\\
Observe that, since we don't have $=$, $m^{'}\models F$.\\
Contradiction. (since $|D_{m^{'}}|=3$)\\
\textbf{16.6}: \\
This is a tricky problem. It is quite tempting to try $\forall$-elim $n$ times and substitute the appropriate terms (wrt $\sigma$)\\
However, this approach fails as the following example shows:\\       
Consider the formula $\forall x_1\text{,}x_2\text{,}x_3\:\: x_1=x_3$ and let $\sigma= [x_1\rightarrow x_2,x_2\rightarrow x_1]$\\
If we try to apply $\forall$-elim starting from $x_1$ and substituting each term (according to $\sigma$), we get the formula $x_1=x_3$.\\
However, $F\sigma= \:(x_2=x_3)$  (As $x_2$ doesn't occur in $F$)\\
A possible solution to this issue is the following: \\
We use extra variables $y_1,y_2\dots y_n$  ($y_i$ are fresh variables not in $F$ and distinct from $x_i$)\\
If $\sigma$ maps $x_i$ to $t_i$, we replace all occurences of $x_i$ in $t_i$ with $y_i$ to create a new term $t_i^{'}$\\
We apply $\forall$-elim on the $x_i$'s and replace with the corresponding terms $t_i^{'}$\\
We obtain a formula $F^{'}$ which has some $y_i$'s occuring in it (we want to replace these $y_i$'s with $x_i$'s to obtain $F\sigma$)\\
Now apply $\forall$-intro for $y_i$'s in order, starting from $y_1$ (Why can we do this?)\\
Now, apply $\forall$-elim again (on the the universally quantified $y_i$'s) and replace each term with the corresponding $x_i$ to get $F\sigma$ (Check that $\forall$-elim can be applied in this context)\\
With this, we get a proof of atmost $2n$ steps\\
\textbf{16.8}: \\
We will simulate $\forall$-elim using $\exists$-def and other proof rules\\
$t$ refers to any arbitrary term \\
\begin{enumerate}
    \item $\Sigma \vdash \forall x F(x)$ \hspace{2mm} (Premise)
    \item $\Sigma \vdash \neg \exists x (\neg F(x))$ \hspace{2mm} ($\exists$-def-1)
    \item $\Sigma \cup \{{\neg F(t)}\} \vdash \neg F(t)$ \hspace{2mm} (Associativity)
    \item $\Sigma \cup \{{\neg F(t)}\} \vdash \exists x (\neg F(x))$ \hspace{2mm} ($\exists$-intro-3)
    \item $\Sigma \cup \{{\neg F(t)}\} \vdash \neg \exists x(\neg F(x))$ \hspace{2mm} (Monotonicity-2)
    \item $\Sigma \vdash \neg \neg F(t)$  \hspace{2mm} (By-Contra-4-5)
    \item  $\Sigma \vdash F(t)$  \hspace{2mm} (Double-neg-elim-6)
\end{enumerate}
\textbf{18.10}: \\
We are trying to prenex a formula $F$ such that the sum of function parameters after skolemization is minimum.\\
First, we remove all occurences of $\implies$ from $F$. This can be done in linear time.\\
We end up a with a formula $F^{'}$ which has only $\vee$ and $\wedge$ as its binary connectives.\\
Now, the main idea is to divide $F^{'}$ into 'chunks' i.e divide $F^{'}$ into separetely quantified blocks,\\
where we call each block a 'chunk'\\
For example, if $F^{'}= (\forall x,y\exists w(R(x,y,z)))\vee (\exists a,b\forall c(E(a,b,c)))$, then $F^{'}$ has two chunks.\\
These are $\forall x,y\exists w(R(x,y,z))$ and $\exists a,b\forall c(E(a,b,c))$.\\
We associate a sequence to each chunk in $F^{'}$. This is done by breaking each chunk into 'pieces'.\\
For each chunk, go through its quantifiers such that whenever a $\exists$ is encountered (or the end is reached), break that part and call it a 'piece'\\
In the example above, in the first chunk of $F^{'}$ we have only one piece i.e $\forall x,y\exists w$\\
while in the second chunk, we have three pieces $\exists a$, $\exists b$ and $\forall c$.\\
Note that if there are no $\exists$ occuring in the chunk, then the $\forall$s form one single piece.\\
For each piece in a given chunk, the value of that piece is the number of $\forall$'s occuring in it\\
The set of values of the pieces naturally generate a sequence for each chunk of the formula\\
In the example, the sequences are $2$ and $0 \:0\:1$.\\
The algorithm goes as follows: \\
\begin{enumerate}
\item Store the sequences in separate linked lists with the pointer to the head of each list stored in another list (call it $L$, $L$ is doubly linked to make deletions easier).\\
\item Compare the values at the head of each linked list (while going through $L$) and prenex the chunk with the minimum value and move the head of that list (with the minimum) to the list$\rightarrow$ next. (if list $\rightarrow$ next $\neq$ NULL)
      If list $\rightarrow$ next $=$ NULL, delete the pointer to the head of that list from $L$
\item Stop iterating when $L$ is empty
\end{enumerate}
We obtain a prenexed formula $G$ which we can guarantee will produce minimal number of parameters after skolemization.\\
\end{flushleft}
\end{document}