\chapter*{Tutorial-1}
\addcontentsline{toc}{chapter}{Tutorial-1}
\textbf{1}:
Given $\phi= \forall x\exists y \:R(x,y)\: \wedge\: \exists y \forall x\:\neg R(x,y)$.
Take the model $m$ to be the set of naturals $\mathbb{N}$ with $<$ relation.
Then, $m\models \phi$ (Why?)\\
(\textbf{Hint}: $\mathbb{N}$ is an well-ordered set (i.e has a minimum) and isn't bounded above)\\
\textbf{2}:
$\varphi_B(x,y)=\exists z(P(z,x)\wedge P(z,y))\wedge \neg F(x)$\\
$\varphi_A(x,y)=\exists z(P(z,y)\wedge \varphi_S(x,z))$   \hspace{4.5mm} ($\varphi_S(x,y)=\exists z(P(z,x)\wedge P(z,y))\wedge F(x)$, $x$ is sister of $y$)\\
$\varphi_C(x,y)=\exists z(\varphi_A(z,x)\wedge P(z,y))$\\
$\varphi_O(x)=\forall z,y (P(z,y)\wedge P(z,x)\Rightarrow (x=y))$\\
The spousal relationship cannot be defined (Why?)\\
\textbf{3}:
$Zero(x)=\:+(x,x)=x$\\
$One(x)= \forall y(\times(x,y)=y)$\\
$Two(x)= \exists z((+(z,z)=x)\wedge One(z))$\\
$Even(x)= \exists z,y((\times(z,y)=x)\wedge Two(y))$ \\
$Odd(x)=\neg Even(x)$\\
$Prime(x)= \neg One(x)\wedge (\neg \exists w,y((\times(w,y)=x)\wedge(\neg One(w)\wedge \neg One(y))))$\\
Goldbach conjecture in FO: $\forall x(\neg One(x)\wedge \neg Two(x)\wedge Even(x)\Rightarrow \exists z,w(Prime(z)\wedge Prime(w)\wedge +(z,w)=x))$\\
\textbf{4}:
Encoding associativity of $+$:  $\forall x,y,z(+(x,+(y,z))=+(+(x,y),z))$\\
Encoding the right identity as $0$:  $\forall x(+(x,0)=x)$\\
Encoding right inverse: $\forall x\exists y(+(x,y)=0)$\\
Encoding A(4): $\forall x,y,z(+(x,z)=+(y,z)\Rightarrow x=y)$\\
Here we have used the signature $\tau=(0,+)$.\\
\textbf{5}:
\textbf{(i)} Consider the set of integers $\mathbb{Z}$ with the induced relation $+_Z$ referring to the usual addition in $\mathbb{Z}$. The constant $0_Z$ refers to $0$ in $\mathbb{Z}$. Observe that addition is associative and admits both left and right inverses. Also $0_{\mathbb{Z}}$ is a identity for addition.
We can conclude the $\tau$-structure $\mathbb{Z}$ satisfies $\psi$.\\
\textbf{(ii)} Consider the set $\mathbb{N}_0$ of whole numbers and the corresponding induced relation being addition and the constant being $0$ (in $\mathbb{N}_0$). This $\tau$-structure doesn't satisfy $\psi$ as $\varphi_3$ fails to be true (non-zero elements in $\mathbb{N}_0$ don't have inverses).\\
\textbf{(iii)} Consider the set of all $n\times n$ invertible matrices with complex values, $GL_n(\mathbb{C})$. Let the induced binary operation be matrix multiplication and let the constant $0$ map to the identity $n\times n$ matrix.\\
It's clear that the $\tau$-structure $GL_n(\mathbb{C})$ satisfies $\psi$, however, it doesn't satisfy $\forall x,y(+(x,y)=+(y,x))$ (Why?).\\
\textbf{(iv)} As before, consider the set $\mathbb{N}_0$ of whole numbers with the usual addition. This satisfies $\varphi_1\wedge\varphi_2$ but doesn't satisfy $\varphi_3$.\\
Consider the set of non-negative reals $\mathbb{R}_{\geq0}$, with the binary operation defined as $+(a,b)=|a-b|$ and the constant mapping to $0$. Show that this structure satisfies $\varphi_2\wedge \varphi_3$ but fails to satisfy $\varphi_1$.\\
Consider $\mathbb{Z}$ with the usual addition and the constant $0$ mapping to $1$ (in $\mathbb{Z}$). This satisfies $\varphi_1\wedge\varphi_3$ but fails to satisfy $\varphi_2$.\\
We can conclude that $\psi$ isn't equivalent to any of $\varphi_1\wedge\varphi_2$, $\varphi_2\wedge\varphi_3$ or $\varphi_1\wedge\varphi_3$.

\textbf{7}:

\tikzset{every picture/.style={line width=0.75pt}} %set default line width to 0.75pt        

\begin{tikzpicture}[x=0.75pt,y=0.75pt,yscale=-1,xscale=1]
%uncomment if require: \path (0,300); %set diagram left start at 0, and has height of 300

%Shape: Circle [id:dp8408479190255029] 
\draw   (100,138) .. controls (100,124.19) and (111.19,113) .. (125,113) .. controls (138.81,113) and (150,124.19) .. (150,138) .. controls (150,151.81) and (138.81,163) .. (125,163) .. controls (111.19,163) and (100,151.81) .. (100,138) -- cycle ;
%Shape: Circle [id:dp750067727127359] 
\draw   (290,59) .. controls (290,45.19) and (301.19,34) .. (315,34) .. controls (328.81,34) and (340,45.19) .. (340,59) .. controls (340,72.81) and (328.81,84) .. (315,84) .. controls (301.19,84) and (290,72.81) .. (290,59) -- cycle ;
%Shape: Circle [id:dp24092758810894677] 
\draw   (292.33,227.17) .. controls (292.33,213.45) and (303.45,202.33) .. (317.17,202.33) .. controls (330.88,202.33) and (342,213.45) .. (342,227.17) .. controls (342,240.88) and (330.88,252) .. (317.17,252) .. controls (303.45,252) and (292.33,240.88) .. (292.33,227.17) -- cycle ;
%Straight Lines [id:da33436515075759754] 
\draw    (148.33,147) -- (294.33,217) ;
%Straight Lines [id:da7853123539498421] 
\draw    (149.33,129) -- (290,59) ;
%Curve Lines [id:da8590354572311187] 
\draw    (102.33,127) .. controls (43.33,100) and (41.33,176) .. (103.33,153) ;
%Curve Lines [id:da3623243093402584] 
\draw    (338.33,47) .. controls (390.33,30) and (410.33,89) .. (336.33,72) ;
%Curve Lines [id:da6684122826682815] 
\draw    (338.33,215) .. controls (378.33,185) and (420.33,248) .. (340.33,241) ;

% Text Node
\draw (120,132) node [anchor=north west][inner sep=0.75pt]   [align=left] {A};
% Text Node
\draw (311,52) node [anchor=north west][inner sep=0.75pt]   [align=left] {B};
% Text Node
\draw (311,220) node [anchor=north west][inner sep=0.75pt]   [align=left] {C};


\end{tikzpicture}\\

\vspace{3mm}
Consider the undirected graph $\mathcal{G}$ above (with loops). This (with its natural edge relation) satisfies the second formula but not the first.\\
\textbf{8}: $\exists^{\geq n}x(x=x)\:\wedge\:\neg\exists ^{\geq n+1}x(x=x)$ is true for all models whose universe has exactly $n$ elements.\\
Let $\varphi= \exists x_1, x_2\dots x_n(\wedge^{}_{i\neq j}(x_i\neq x_j))$.\\
$\varphi \equiv \exists^{\geq n}x(x=x)$ (Why?)\\
\textbf{9}:
Using counting quantifiers, we can write,\\
$\varphi= \exists^{\geq n}x(x=x)\:\wedge\:\neg\exists ^{\geq m+1}x(x=x)$\\
$\varphi$ evaluates to true only over models with atleast $n$ and atmost $m$ elements.